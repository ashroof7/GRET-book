
%----------------------------------------------------------------------------------------
%	CHAPTER 0 // Preface
%----------------------------------------------------------------------------------------

\chapterimage{cover1.jpg} % Chapter heading image
\chapter*{Preface}
\section*{Abstract}\index{Abstract}
With the ever-increasing diffusion of wearable computers in our lives, and the increasing time we spend using such devices, developing new techniques that ease interaction with computers has become insistent. One of the most interesting topics in this field is how to allow user to interact with computers without the traditional mode of interaction (mouse, keyboard and even touch-screen). We believe that eye tracking and gesture recognition seem to be very appealing technologies to achieve this goal.\bigskip

Wearable gadgets like Google Glass is a promising example of ubiquitous computers that might be a an essential part of our lifestyle in the near future. Glass displays information in a smartphone-like hands-free format, that can communicate with the Internet via natural language voice commands. In this project, two new glass interaction techniques are proposed namely; eye tracking and vision-based gesture recognition.\bigskip

Eye motion tracking serves as powerful tool to know the user's point of gaze and attentional state. Hence this information is used to increase the responsiveness of the computer in respond to users actions. Moreover eye-motions can be translated into commands for Glass. Another feature of eye tracking is that it can be used to provide aid to people with disabilities hindering them from casual interaction with wearable gadgets.\bigskip

Gesture recognition has the potential to be a natural and powerful tool supporting efficient and intuitive interaction between the human and the computer (Glass). Visual interpretation of hand gestures can be interpreted into commands for Glass which definitely will help in achieving the ease and naturalness desired for Human Computer Interaction (HCI). Interpreting sign language is a very promising application of gesture recognition, offering an easy alternative way of interaction that will help many deaf people.